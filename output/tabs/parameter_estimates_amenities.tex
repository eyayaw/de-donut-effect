
\begin{table}
   \centering
   \begin{threeparttable}[b]
      
      \bigskip
      \begin{tabular}{lcccc}
         \toprule
         Type & \multicolumn{2}{c}{\textbf{Consumption amenities}} & \multicolumn{2}{c}{\textbf{Environmental amenities}} \\ \cmidrule(lr){2-3} \cmidrule(lr){4-5}
                             & $\ln \text{count}$   & \text{index}    & $\ln\text{area km}^2$   & \text{index}\\   
                             & (1)                  & (2)             & (3)                     & (4)\\  
         \midrule 
         Constant            & 4.323$^{***}$        & 0.9777$^{***}$  & -6.538$^{***}$          & 0.0169\\   
                             & (0.0117)             & (0.0508)        & (0.0320)                & (0.0485)\\   
         $\ln \text{dist}$   & -0.2468$^{***}$      & -0.3508$^{***}$ & 0.1669$^{***}$          & -0.0063\\   
                             & (0.0050)             & (0.0166)        & (0.0132)                & (0.0183)\\   
          \\
         Observations        & 65,830               & 8,478           & 65,830                  & 3,800\\  
         R$^2$               & 0.03564              & 0.08925         & 0.00231                 & $3.13\times 10^{-5}$\\   
         \bottomrule
      \end{tabular}
      
      \begin{tablenotes}\item \textit{Notes:} The table shows the estimates for the parameters $b$ and $\phi$ from Equation (\ref{eq-amenities}), which represent the relationship between amenities and the distance to the CBD. The estimation uses amenity data for 2019. The number of observations for the indices represents zip codes. In the raw data, the number of observations counts the amenity object.
      \end{tablenotes}
   \end{threeparttable}
\end{table}


